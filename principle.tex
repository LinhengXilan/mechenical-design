\documentclass[a4paper]{article}
\usepackage[UTF8]{ctex}
\usepackage{geometry}
\usepackage{setspace}
\usepackage{amsmath}
\usepackage{cases}
\usepackage{amssymb}
\usepackage{color}
\usepackage[rgb]{xcolor}
\setstretch{1.5}
\geometry{a4paper, top=1.8cm, left=2.5cm, right=2.5cm, bottom=1.8cm}


\newcommand{\ColorImportant}{\textcolor{red!100!green!80!blue!60}}
\newcommand{\ColorFormula}{\textcolor{red!100!green!73!blue!100}}


\title{机械原理}
\author{lhxl}


\begin{document}
\maketitle


\raggedright
\section{机构组成和结构分析}
\subsection{机构的组成}
\subsubsection{构件}
构件是参与运动的最小单元,零件是单独加工的最小单元。\\

\subsubsection{运动副}
运动副是构件间接触形成的可动连接。\\
构建未连接时由六个自由度。\\
引入n个约束的运动副称为n级副。\\
面接触的为低副,其他为高副,高副比低副更易磨损。\\
常见运动副符号见【P10~P12,表1.1】。\\

\subsubsection{运动链}
运动链是多个构件通过运动副构成的系统。\\
如果运动链首末封闭则为闭链,否则为开链。\\

\subsubsection{机构}
在运动链中若固定某一构件,使其他构件有确定的相对运动,则此运动链成为机构。\\



\subsection{机构运动简图}
\subsubsection{运动简图}
见【P13~P15,表1.2;P15~P17,表1.3】。\\

\subsubsection{运动简图的绘制}
见书。\\


\subsection{运动链成为机构的条件}
\subsubsection{运动链的自由度计算}
设有$n$个构件,则初始有$6n$个自由度。\\
当引入$p_n$个$n$级副时。\\
自由度为:\\
\centering$F=6n-5p_5-4p_4-3p_3-2p_2-p_1$\\
\raggedright 平面运动链中\\
\centering$F=3n-2p_5-p_4$\\

\raggedright\subsubsection{运动链成为机构的条件}
运动链的自由度大于零,且原动件的数量等于自由度时,有确定的运动则称为机构。\\

\subsubsection{计算自由度时应注意的问题}
1. 复合铰链\\
若$k$个构件在同一点构成运动副,则运动副数量为$k-1$。\\
2. 局部自由度\\
有些构件产生的自由度并不影响其他构件,仅在自身局部产生作用。\\
因此计算自由度时应减去。\\
\centering$F=3n-2p_5-p_4-F_{局部}$\\
\raggedright 3. 虚约束\\
1. 自由度被多个运动副同时限制。\\
2. 不同自由度产生同样的效果。\\
3. 不同构件运动轨迹相同的。\\
4. 约束不影响运动的。\\
这些多余的约束称为虚约束,计算自由度时应忽略。\\


\subsection{机构的组成原理和结构分析}
\subsubsection{平面机构的高副低代}
高副低代的条件:\\
1. 代替后自由度不变。\\
2. 代替后速度和加速度不变。\\

\subsubsection{机构的组成原理}
1. 杆组\\
杆组是机构中不可再分的自由度为零的构建组合。\\
\centering$3n-2p_5=0$\\
\raggedright 因此n只能为偶数。\\
常见杆组见【P25,图1.20、图1.21】。\\
\vspace*{1em}
2. 机构的组成原理\\
把若干个自由度为零的基本杆组连接到机架和原动件上。\\
满足条件的情况下,机构越简单越好。\\

\subsubsection{机构的结构分析}
见书。\\



\section{连杆机构}
\subsection{平面连杆机构的类型}
结构最简单运用最广泛的是平面四杆机构。\\

\subsubsection{平面四杆机构的基本型式}
非机架的杆件称作摇杆,能做整周运动的称为曲柄。\\
\vspace*{1em}
1. 曲柄摇杆机构\\
2. 双曲柄机构\\
两曲柄长度相同时称为平行四边形机构,可能会卡死,因此可在从动曲柄上装一个惯性较大的轮子。\\
3. 双摇杆机构\\
两曲柄长度相同时称为等腰梯形机构。\\

\subsubsection{平面四杆机构的演化}
1. 转动副转化成移动副。\\
2. 选取不同构件为机架。\\
3. 变换构件的形态。\\
4. 扩大转动副的尺寸。\\


\subsection{平面连杆机构的工作特性}
\subsubsection{运动特性}
1. 转动副为整转副的条件\\
1)组成整转副的两个构件中必有一个最短杆,且最长和最短杆长度之和必小于等于其余两构件的长度。\\
\vspace*{1em}
2. 急回运动特性\\
两摇杆的平均角速度不同,这种运动称为急回运动。\\
行程速度变化系数:\\
\centering$K=\frac{\varphi_1}{\varphi_2}=\frac{180^\circ+\theta}{180^\circ-\theta}$\\
\raggedright 极位夹角:\\
\centering$\theta=180^\circ\frac{K-1}{K+1}$\\
\raggedright $\theta$角和$K$越大,急回运动特性越显著。\\
\vspace*{1em}
3. 运动的连续性\\
构件的各自的可行域必须连续,否则运动不连续。\\

\subsubsection{传力特性}
1. 压力角和传动角\\
$$\begin{cases}
    F_t=&F\cos\alpha \\
    F_n=&F\sin\alpha
\end{cases}$$
$\alpha$称为压力角。\\
压力角的余角称为传动角:\\
\centering$\gamma=90^\circ-\alpha$\\
\raggedright 传动角越大,机构效率越高。\\
为了保证机构有良好传力性,$\gamma\geqslant40^\circ$,对于高速和大功率传动器械,$\gamma\geqslant50^\circ$。\\
\vspace*{1em}
2. 死点位置\\
机构运行中,从动件可能出现共线的情况,此时从动件的传动角为0。\\
应采取措施使机构通过死点。\\
有时死点也能起到夹紧等作用。\\


\subsection{平面连杆机构的的特点及功能}
\subsubsection{平面连杆机构的特点}
1. 构件间以低副连接,可传递较大动力。\\
2. 构建运动具有多样性。\\
3. 再主动件运动规律不变的情况下,只要改变连杆机构各构件的相对尺寸,就可以使从动件实现不同的运动规律的运动要求。\\
4. 连杆曲线具有多样性。\\
5. 在连杆机构的运动过程中,一些构件的质心在做变速运动,产生动载荷,发生震动,因此连杆机构不适于高速运动。\\
6. 由于各构件尺寸不可能完全精准,因此运动误差较大。\\

\subsubsection{平面连杆机构的功能}
1. 实现有轨迹、位置或有运动规律要求的运动。\\
2. 实现从动件运动形式及运动特性的改变。\\
3. 实现较远距离的传动。\\
4. 调节、扩大从动件行程。\\
5. 获得较大的机械增益。\\
机械输出力矩(或力)与输入力矩(或力)的比值称为机械增益。\\


\subsection{平面连杆机构的运动分析}
\subsubsection{瞬心法及其应用}
1. 速度瞬心\\
当两构件做平面相对运动时,在任意瞬间,都可以认为他们在绕某一重合点作相对运动,这个点被称为速度瞬心。\\
\vspace*{1em}
2. 机构中速度瞬心的数量\\
\centering$N=n(n-1)/2$\\
\vspace*{1em}
\raggedright 3. 机构中瞬心位置的确定\\
1)通过运动副连接的构件\\
(1)以转动副连接的构件\\
运动副的连接处即为瞬心。\\
(2)以移动副链接的两构件的瞬心\\
瞬心位于垂直于移动方向的无限远处。\\
(3)以平面高副连接的两构件的瞬心\\
若运动为纯滚动,则瞬心为滚动接触点。\\
若运动为滚动加运动,则瞬心位于转动切向方向。\\
2)不直接相连的两构件的瞬心\\
此瞬心满足三心定理:做平面运动的三个构件的三个瞬心位于同一直线。\\
\vspace*{1em}
4. 瞬心在速度分析中的应用\\
见【P53,图2.48】\\
可看作AB和CD以$P_{24}$为圆心转动。\\

\subsubsection{平面机构的整体运动分析法}
见书。\\
二次方程求根公式中根号项M前为位置模式。\\

\subsubsection{基本杆组法及其应用}
1. 基本杆组法\\
2. 基本杆组法的应用\\

\subsection{平面连杆机构的运动设计}
\subsubsection{平面连杆机构设计的基本问题}
1. 实现刚体给定位置的设计\\
2. 实现预定运动规律的设计\\
3. 事先预定轨迹的设计\\

\subsubsection{刚体导引机构的设计}
见书。\\

\subsubsection{函数生成机构的设计}
略。\\

\subsubsection{急回机构的设计}
通常要求按照给定的K值设计。\\
首先计算极位夹角。\\
作图法见书。\\

\subsubsection{轨迹生成机构的设计}
略。\\


\subsection{空间连杆机构}
略。\\



\section{凸轮机构}
\subsection{凸轮机构的组成和特点}
\subsubsection{凸轮机构的组成}
凸轮机构由凸轮、从动件、机架组成。\\

\subsubsection{凸轮机构的类型}
1. 按照凸轮形状分\\
1)盘型凸轮、2)移动凸轮、3)圆柱凸轮。\\
\vspace*{1em}
2. 按照从动件的形状分\\
1)尖端从动件\\
能与任意复杂的凸轮配合使用。\\
结构简单,尖端处易磨损。\\
2)曲面从动件\\
应用较多。\\
3)滚子从动件\\
从动件与凸轮之间有滑动摩擦。\\
4)平底从动件\\
从动件与凸轮之间为线接触,润滑好。\\
凸轮对从动件的作用力永远垂直于底面,受力平稳,传动效率高,常用于高速场合。\\
缺点是凸轮必须是外凸形状。\\
\vspace*{1em}
3. 按照从动件运动类型分类\\
1)移动从动件,2)摆动从动件。\\
\vspace*{1em}
4. 按照凸轮与从动件维持高副接触的方法分类\\
1)力封闭型凸轮机构\\
通过重力等维持凸轮与从动件之间的接触。\\
2)型封闭凸轮结构\\
(1)槽凸轮机构,(2)等宽凸轮机构等。。。\\


\subsection{凸轮机构的特点和功能}
\subsubsection{凸轮机构的特点}
结构简单紧凑。\\
%\\\\\\\\\\\\\\\\\\\\\\\\\\\\\\\\\\\\\\\\\\\\\\\\\\\\\\\\\\\\\\\\\\\\\\\\\\\\\\\\\\\\\\\\\\\\\\\\\\\\\\\\\\\\\\\\\\\\\\\\\\\\\\\\\\\\\\\\\\\\\\\\\\\\\\\\\\\\\\\\\\\\\\\\\\\\\\\\\\\\\\\\\\\\\\\\\\\\\\\\\\\\\\\\\\\\\\\\\\\\\\\\\\\\\\\
%\centering$$\\\raggedright\\\raggedright\\\centering$$\\\raggedright\\\centering$$\\\raggedright\\\centering$$\\\raggedright\\\centering$$\\\raggedright\\\centering$$\\\raggedright\\\centering$$\\\raggedright\\\centering$$\\\raggedright\\\centering$$\\\raggedright\\\centering$$\\\raggedright\\\raggedright\\\centering$$\\\raggedright\\\centering$$\\\raggedright\\\centering$$\\\raggedright\\\centering$$\\\raggedright\\\centering$$\\\raggedright\\\centering$$\\\raggedright\\\centering$$\\\raggedright\\\centering$$\\\raggedright\\\centering$$\\\raggedright\\\raggedright\\\centering$$\\\raggedright\\\centering$$\\\raggedright\\\centering$$\\\raggedright\\\centering$$\\\raggedright\\\centering$$\\\raggedright\\\centering$$\\\raggedright\\\centering$$\\\raggedright\\\centering$$\\\raggedright\\\centering$$\\\raggedright\\\raggedright\\\centering$$\\\raggedright\\\centering$$\\\raggedright\\\centering$$\\\raggedright\\\centering$$\\\raggedright\\\centering$$\\\raggedright\\\centering$$\\\raggedright\\\centering$$\\\raggedright\\\centering$$\\\raggedright\\\centering$$\\\raggedright\\\raggedright\\\centering$$\\\raggedright\\\centering$$\\\raggedright\\\centering$$\\\raggedright\\\centering$$\\\raggedright\\\centering$$\\\raggedright\\\centering$$\\\raggedright\\\centering$$\\\raggedright\\\centering$$\\\raggedright\\\centering$$\\\raggedright\\\raggedright\\\centering$$\\\raggedright\\\centering$$\\\raggedright\\\centering$$\\\raggedright\\\centering$$\\\raggedright\\\centering$$\\\raggedright\\\centering$$\\\raggedright\\\centering$$\\\raggedright\\\centering$$\\\raggedright\\\centering$$\\\raggedright\\\raggedright\\\centering$$\\\raggedright\\\centering$$\\\raggedright\\\centering$$\\\raggedright\\\centering$$\\\raggedright\\\centering$$\\\raggedright\\\centering$$\\\raggedright\\\centering$$\\\raggedright\\\centering$$\\\raggedright\\\centering$$\\\raggedright\\\raggedright\\\centering$$\\\raggedright\\\centering$$\\\raggedright\\\centering$$\\\raggedright\\\centering$$\\\raggedright\\\centering$$\\\raggedright\\\centering$$\\\raggedright\\\centering$$\\\raggedright\\\centering$$\\\raggedright\\\centering$$\\\raggedright\\\raggedright\\\centering$$\\\raggedright\\\centering$$\\\raggedright\\\centering$$\\\raggedright\\\centering$$\\\raggedright\\\centering$$\\\raggedright\\\centering$$\\\raggedright\\\centering$$\\\raggedright\\\centering$$\\\raggedright\\\centering$$\\\raggedright\\\raggedright\\\centering$$\\\raggedright\\\centering$$\\\raggedright\\\centering$$\\\raggedright\\\centering$$\\\raggedright\\\centering$$\\\raggedright\\\centering$$\\\raggedright\\\centering$$\\\raggedright\\\centering$$\\\raggedright\\\centering$$\\\raggedright\\\raggedright\\\centering$$\\\raggedright\\\centering$$\\\raggedright\\\centering$$\\\raggedright\\\centering$$\\\raggedright\\\centering$$\\\raggedright\\\centering$$\\\raggedright\\\centering$$\\\raggedright\\\centering$$\\\raggedright\\\centering$$\\\raggedright\\\raggedright\\\centering$$\\\raggedright\\\centering$$\\\raggedright\\\centering$$\\\raggedright\\\centering$$\\\raggedright\\\centering$$\\\raggedright\\\centering$$\\\raggedright\\\centering$$\\\raggedright\\\centering$$\\\raggedright\\\centering$$\\\raggedright\\\raggedright\\\centering$$\\\raggedright\\\centering$$\\\raggedright\\\centering$$\\\raggedright\\\centering$$\\\raggedright\\\centering$$\\\raggedright\\\centering$$\\\raggedright\\\centering$$\\\raggedright\\\centering$$\\\raggedright\\\centering$$\\\raggedright\\\raggedright\\\centering$$\\\raggedright\\\centering$$\\\raggedright\\\centering$$\\\raggedright\\\centering$$\\\raggedright\\\centering$$\\\raggedright\\\centering$$\\\raggedright\\\centering$$\\\raggedright\\\centering$$\\\raggedright\\\centering$$\\\raggedright\\\raggedright\\\centering$$\\\raggedright\\\centering$$\\\raggedright\\\centering$$\\\raggedright\\\centering$$\\\raggedright\\\centering$$\\\raggedright\\\centering$$\\\raggedright\\\centering$$\\\raggedright\\\centering$$\\\raggedright\\\centering$$\\\raggedright\\\raggedright\\\centering$$\\\raggedright\\\centering$$\\\raggedright\\\centering$$\\\raggedright\\\centering$$\\\raggedright\\\centering$$\\\raggedright\\\centering$$\\\raggedright\\\centering$$\\\raggedright\\\centering$$\\\raggedright\\\centering$$\\\raggedright\\\raggedright\\\centering$$\\\raggedright\\\centering$$\\\raggedright\\\centering$$\\\raggedright\\\centering$$\\\raggedright\\\centering$$\\\raggedright\\\centering$$\\\raggedright\\\centering$$\\\raggedright\\\centering$$\\\raggedright\\\centering$$\\\raggedright\\\raggedright\\\centering$$\\\raggedright\\\centering$$\\\raggedright\\\centering$$\\\raggedright\\\centering$$\\\raggedright\\\centering$$\\\raggedright\\\centering$$\\\raggedright\\\centering$$\\\raggedright\\\centering$$\\\raggedright\\\centering$$\\\raggedright\\\raggedright\\\centering$$\\\raggedright\\\centering$$\\\raggedright\\\centering$$\\\raggedright\\\centering$$\\\raggedright\\\centering$$\\\raggedright\\\centering$$\\\raggedright\\\centering$$\\\raggedright\\\centering$$\\\raggedright\\\centering$$\\\raggedright\\\raggedright\\\centering$$\\\raggedright\\\centering$$\\\raggedright\\\centering$$\\\raggedright\\\centering$$\\\raggedright\\\centering$$\\\raggedright\\\centering$$\\\raggedright\\\centering$$\\\raggedright\\\centering$$\\\raggedright\\\centering$$\\\raggedright\\\raggedright\\\centering$$\\\raggedright\\\centering$$\\\raggedright\\\centering$$\\\raggedright\\\centering$$\\\raggedright\\\centering$$\\\raggedright\\\centering$$\\\raggedright\\\centering$$\\\raggedright\\\centering$$\\\raggedright\\\centering$$\\\raggedright\\\raggedright\\\centering$$\\\raggedright\\\centering$$\\\raggedright\\\centering$$\\\raggedright\\\centering$$\\\raggedright\\\centering$$\\\raggedright\\\centering$$\\\raggedright\\\centering$$\\\raggedright\\\centering$$\\\raggedright\\\centering$$\\\raggedright\\\raggedright\\\centering$$\\\raggedright\\\centering$$\\\raggedright\\\centering$$\\\raggedright\\\centering$$\\\raggedright\\\centering$$\\\raggedright\\\centering$$\\\raggedright\\\centering$$\\\raggedright\\\centering$$\\\raggedright\\\centering$$\\\raggedright\\\raggedright\\\centering$$\\\raggedright\\\centering$$\\\raggedright\\\centering$$\\\raggedright\\\centering$$\\\raggedright\\\centering$$\\\raggedright\\\centering$$\\\raggedright\\\centering$$\\\raggedright\\\centering$$\\\raggedright\\\centering$$\\\raggedright\\\raggedright\\\centering$$\\\raggedright\\\centering$$\\\raggedright\\\centering$$\\\raggedright\\\centering$$\\\raggedright\\\centering$$\\\raggedright\\\centering$$\\\raggedright\\\centering$$\\\raggedright\\\centering$$\\\raggedright\\\centering$$\\\raggedright\\\raggedright\\\centering$$\\\raggedright\\\centering$$\\\raggedright\\\centering$$\\\raggedright\\\centering$$\\\raggedright\\\centering$$\\\raggedright\\\centering$$\\\raggedright\\\centering$$\\\raggedright\\\centering$$\\\raggedright\\\centering$$\\\raggedright\\\raggedright\\\centering$$\\\raggedright\\\centering$$\\\raggedright\\\centering$$\\\raggedright\\\centering$$\\\raggedright\\\centering$$\\\raggedright\\\centering$$\\\raggedright\\\centering$$\\\raggedright\\\centering$$\\\raggedright\\\centering$$\\\raggedright\\\raggedright\\\centering$$\\\raggedright\\\centering$$\\\raggedright\\\centering$$\\\raggedright\\\centering$$\\\raggedright\\\centering$$\\\raggedright\\\centering$$\\\raggedright\\\centering$$\\\raggedright\\\centering$$\\\raggedright\\\centering$$\\\raggedright\\\raggedright\\\centering$$\\\raggedright\\\centering$$\\\raggedright\\\centering$$\\\raggedright\\\centering$$\\\raggedright\\\centering$$\\\raggedright\\\centering$$\\\raggedright\\\centering$$\\\raggedright\\\centering$$\\\raggedright\\\centering$$\\\raggedright\\\raggedright\\\centering$$\\\raggedright\\\centering$$\\\raggedright\\\centering$$\\\raggedright\\\centering$$\\\raggedright\\\centering$$\\\raggedright\\\centering$$\\\raggedright\\\centering$$\\\raggedright\\\centering$$\\\raggedright\\\centering$$\\\raggedright\\\raggedright\\\centering$$\\\raggedright\\\centering$$\\\raggedright\\\centering$$\\\raggedright\\\centering$$\\\raggedright\\\centering$$\\\raggedright\\\centering$$\\\raggedright\\\centering$$\\\raggedright\\\centering$$\\\raggedright\\\centering$$\\\raggedright\\\raggedright\\\centering$$\\\raggedright\\\centering$$\\\raggedright\\\centering$$\\\raggedright\\\centering$$\\\raggedright\\\centering$$\\\raggedright\\\centering$$\\\raggedright\\\centering$$\\\raggedright\\\centering$$\\\raggedright\\\centering$$\\\raggedright\\\raggedright\\\centering$$\\\raggedright\\\centering$$\\\raggedright\\\centering$$\\\raggedright\\\centering$$\\\raggedright\\\centering$$\\\raggedright\\\centering$$\\\raggedright\\\centering$$\\\raggedright\\\centering$$\\\raggedright\\\centering$$\\\raggedright\\\raggedright\\\centering$$\\\raggedright\\\centering$$\\\raggedright\\\centering$$\\\raggedright\\\centering$$\\\raggedright\\\centering$$\\\raggedright\\\centering$$\\\raggedright\\\centering$$\\\raggedright\\\centering$$\\\raggedright\\\centering$$\\\raggedright\\\raggedright\\\centering$$\\\raggedright\\\centering$$\\\raggedright\\\centering$$\\\raggedright\\\centering$$\\\raggedright\\\centering$$\\\raggedright\\\centering$$\\\raggedright\\\centering$$\\\raggedright\\\centering$$\\\raggedright\\\centering$$\\\raggedright\\\raggedright\\\centering$$\\\raggedright\\\centering$$\\\raggedright\\\centering$$\\\raggedright\\\centering$$\\\raggedright\\\centering$$\\\raggedright\\\centering$$\\\raggedright\\\centering$$\\\raggedright\\\centering$$\\\raggedright\\\centering$$\\\raggedright\\\raggedright\\\centering$$\\\raggedright\\\centering$$\\\raggedright\\\centering$$\\\raggedright\\\centering$$\\\raggedright\\\centering$$\\\raggedright\\\centering$$\\\raggedright\\\centering$$\\\raggedright\\\centering$$\\\raggedright\\\centering$$\\\raggedright\\\raggedright\\\centering$$\\\raggedright\\\centering$$\\\raggedright\\\centering$$\\\raggedright\\\centering$$\\\raggedright\\\centering$$\\\raggedright\\\centering$$\\\raggedright\\\centering$$\\\raggedright\\\centering$$\\\raggedright\\\centering$$\\\raggedright\\\raggedright\\\centering$$\\\raggedright\\\centering$$\\\raggedright\\\centering$$\\\raggedright\\\centering$$\\\raggedright\\\centering$$\\\raggedright\\\centering$$\\\raggedright\\\centering$$\\\raggedright\\\centering$$\\\raggedright\\\centering$$\\\raggedright\\\raggedright\\\centering$$\\\raggedright\\\centering$$\\\raggedright\\\centering$$\\\raggedright\\\centering$$\\\raggedright\\\centering$$\\\raggedright\\\centering$$\\\raggedright\\\centering$$\\\raggedright\\\centering$$\\\raggedright\\\centering$$\\\raggedright\\\raggedright\\\centering$$\\\raggedright\\\centering$$\\\raggedright\\\centering$$\\\raggedright\\\centering$$\\\raggedright\\\centering$$\\\raggedright\\\centering$$\\\raggedright\\\centering$$\\\raggedright\\\centering$$\\\raggedright\\\centering$$\\\raggedright\\\raggedright\\\centering$$\\\raggedright\\\centering$$\\\raggedright\\\centering$$\\\raggedright\\\centering$$\\\raggedright\\\centering$$\\\raggedright\\\centering$$\\\raggedright\\\centering$$\\\raggedright\\\centering$$\\\raggedright\\\centering$$\\\raggedright\\\raggedright\\\centering$$\\\raggedright\\\centering$$\\\raggedright\\\centering$$\\\raggedright\\\centering$$\\\raggedright\\\centering$$\\\raggedright\\\centering$$\\\raggedright\\\centering$$\\\raggedright\\\centering$$\\\raggedright\\\centering$$\\\raggedright\\\raggedright\\\centering$$\\\raggedright\\\centering$$\\\raggedright\\\centering$$\\\raggedright\\\centering$$\\\raggedright\\\centering$$\\\raggedright\\\centering$$\\\raggedright\\\centering$$\\\raggedright\\\centering$$\\\raggedright\\\centering$$\\\raggedright\\\raggedright\\\centering$$\\\raggedright\\\centering$$\\\raggedright\\\centering$$\\\raggedright\\\centering$$\\\raggedright\\\centering$$\\\raggedright\\\centering$$\\\raggedright\\\centering$$\\\raggedright\\\centering$$\\\raggedright\\\centering$$\\\raggedright\\\raggedright\\\centering$$\\\raggedright\\\centering$$\\\raggedright\\\centering$$\\\raggedright\\\centering$$\\\raggedright\\\centering$$\\\raggedright\\\centering$$\\\raggedright\\\centering$$\\\raggedright\\\centering$$\\\raggedright\\\centering$$\\\raggedright\\\raggedright\\\centering$$\\\raggedright\\\centering$$\\\raggedright\\\centering$$\\\raggedright\\\centering$$\\\raggedright\\\centering$$\\\raggedright\\\centering$$\\\raggedright\\\centering$$\\\raggedright\\\centering$$\\\raggedright\\\centering$$\\\raggedright\\\raggedright\\\centering$$\\\raggedright\\\centering$$\\\raggedright\\\centering$$\\\raggedright\\\centering$$\\\raggedright\\\centering$$\\\raggedright\\\centering$$\\\raggedright\\\centering$$\\\raggedright\\\centering$$\\\raggedright\\\centering$$\\\raggedright\\\raggedright\\\centering$$\\\raggedright\\\centering$$\\\raggedright\\\centering$$\\\raggedright\\\centering$$\\\raggedright\\\centering$$\\\raggedright\\\centering$$\\\raggedright\\\centering$$\\\raggedright\\\centering$$\\\raggedright\\\centering$$\\\raggedright\\\raggedright\\\centering$$\\\raggedright\\\centering$$\\\raggedright\\\centering$$\\\raggedright\\\centering$$\\\raggedright\\\centering$$\\\raggedright\\\centering$$\\\raggedright\\\centering$$\\\raggedright\\\centering$$\\\raggedright\\\centering$$\\\raggedright\\
\subsubsection{}
\subsubsection{}
\subsubsection{}
\subsubsection{}
\subsubsection{}
\subsubsection{}
\subsubsection{}
\subsubsection{}
\subsubsection{}


\subsection{从动件运动规律设计}
\subsubsection{}
\subsubsection{}
\subsubsection{}
\subsubsection{}
\subsubsection{}
\subsubsection{}
\subsubsection{}
\subsubsection{}
\subsubsection{}
\subsubsection{}


\subsection{凸轮廓线设计}
\subsubsection{}
\subsubsection{}
\subsubsection{}
\subsubsection{}
\subsubsection{}
\subsubsection{}
\subsubsection{}
\subsubsection{}
\subsubsection{}
\subsubsection{}


\subsection{凸轮机构基本参数设计}
\subsubsection{}
\subsubsection{}
\subsubsection{}
\subsubsection{}
\subsubsection{}
\subsubsection{}
\subsubsection{}
\subsubsection{}
\subsubsection{}
\subsubsection{}



\section{齿轮机构}
\subsection{齿轮机构的组成和类型}
\subsubsection{}
\subsubsection{}
\subsubsection{}
\subsubsection{}
\subsubsection{}
\subsubsection{}
\subsubsection{}
\subsubsection{}
\subsubsection{}
\subsubsection{}


\subsection{渐开线齿廓及其啮合特性}
\subsubsection{}
\subsubsection{}
\subsubsection{}
\subsubsection{}
\subsubsection{}
\subsubsection{}
\subsubsection{}
\subsubsection{}
\subsubsection{}
\subsubsection{}


\subsection{渐开线标准直齿圆柱齿轮}
\subsubsection{}
\subsubsection{}
\subsubsection{}
\subsubsection{}
\subsubsection{}
\subsubsection{}
\subsubsection{}
\subsubsection{}
\subsubsection{}
\subsubsection{}


\subsection{渐开线标准直齿圆柱齿轮的啮合传动}
\subsubsection{}
\subsubsection{}
\subsubsection{}
\subsubsection{}
\subsubsection{}
\subsubsection{}
\subsubsection{}
\subsubsection{}
\subsubsection{}
\subsubsection{}


\subsection{渐开线的范成加工及渐开线齿廓的根切}
\subsubsection{}
\subsubsection{}
\subsubsection{}
\subsubsection{}
\subsubsection{}
\subsubsection{}
\subsubsection{}
\subsubsection{}
\subsubsection{}
\subsubsection{}


\subsection{渐开线变位齿轮}
\subsubsection{}
\subsubsection{}
\subsubsection{}
\subsubsection{}
\subsubsection{}
\subsubsection{}
\subsubsection{}
\subsubsection{}
\subsubsection{}
\subsubsection{}


\subsection{渐开线直齿圆柱齿轮的传动设计}
\subsubsection{}
\subsubsection{}
\subsubsection{}
\subsubsection{}
\subsubsection{}
\subsubsection{}
\subsubsection{}
\subsubsection{}
\subsubsection{}
\subsubsection{}


\subsection{渐开线圆柱齿轮机构}
\subsubsection{}
\subsubsection{}
\subsubsection{}
\subsubsection{}
\subsubsection{}
\subsubsection{}
\subsubsection{}
\subsubsection{}
\subsubsection{}
\subsubsection{}


\subsection{蜗杆蜗轮机构}
\subsubsection{}
\subsubsection{}
\subsubsection{}
\subsubsection{}
\subsubsection{}
\subsubsection{}
\subsubsection{}
\subsubsection{}
\subsubsection{}
\subsubsection{}


\subsection{圆锥齿轮机构}
\subsubsection{}
\subsubsection{}
\subsubsection{}
\subsubsection{}
\subsubsection{}
\subsubsection{}
\subsubsection{}
\subsubsection{}
\subsubsection{}
\subsubsection{}


\subsection{非圆齿轮机构}
\subsubsection{}
\subsubsection{}
\subsubsection{}
\subsubsection{}
\subsubsection{}
\subsubsection{}
\subsubsection{}
\subsubsection{}
\subsubsection{}
\subsubsection{}



\section{轮系}
\subsection{轮系的类型}
\subsubsection{}
\subsubsection{}
\subsubsection{}
\subsubsection{}
\subsubsection{}
\subsubsection{}
\subsubsection{}
\subsubsection{}
\subsubsection{}
\subsubsection{}


\subsection{轮系的传动比}
\subsubsection{}
\subsubsection{}
\subsubsection{}
\subsubsection{}
\subsubsection{}
\subsubsection{}
\subsubsection{}
\subsubsection{}
\subsubsection{}
\subsubsection{}


\subsection{轮系的效率}
\subsubsection{}
\subsubsection{}
\subsubsection{}
\subsubsection{}
\subsubsection{}
\subsubsection{}
\subsubsection{}
\subsubsection{}
\subsubsection{}
\subsubsection{}


\subsection{轮系的功能}
\subsubsection{}
\subsubsection{}
\subsubsection{}
\subsubsection{}
\subsubsection{}
\subsubsection{}
\subsubsection{}
\subsubsection{}
\subsubsection{}
\subsubsection{}


\subsection{轮系的设计}
\subsubsection{}
\subsubsection{}
\subsubsection{}
\subsubsection{}
\subsubsection{}
\subsubsection{}
\subsubsection{}
\subsubsection{}
\subsubsection{}
\subsubsection{}


\subsection{行星传动}
\subsubsection{}
\subsubsection{}
\subsubsection{}
\subsubsection{}
\subsubsection{}
\subsubsection{}
\subsubsection{}
\subsubsection{}
\subsubsection{}
\subsubsection{}



\section{平衡}
\subsubsection{}
\subsubsection{}
\subsubsection{}
\subsubsection{}
\subsubsection{}
\subsubsection{}
\subsubsection{}
\subsubsection{}
\subsubsection{}
\subsubsection{}


\end{document}